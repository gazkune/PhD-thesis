
% this file is called up by thesis.tex
% content in this file will be fed into the main document

%: ----------------------- introduction file header -----------------------
\begin{savequote}[50mm]
If you keep doing what you've always done, you'll keep getting what you've always gotten.
\qauthor{Neil Strauss}
\end{savequote}


\chapter{A Clustering Process for Activity Annotation}
\label{cha:clustering}

% the code below specifies where the figures are stored
\ifpdf
    \graphicspath{{4_clustering_process/figures/PDF/}{4_clustering_process/figures/PNG/}{4_clustering_process/figures/}}
\else
    \graphicspath{{4_clustering_process/figures/EPS/}{4_clustering_process/figures/}}
\fi

\letra{A}{} novel clustering process for activity annotation in an unlabelled dataset is described in this chapter. Using previous knowledge obtained from a domain expert, the clustering process identifies activities performed by a user in an unlabelled sensor activation dataset. The clustering process is claimed to be novel because it combines unsupervised learning techniques with domain knowledge. In contrast with a typical clustering algorithm, which finds a structure in a given dataset, the proposed algorithm can also identify the extracted structures with a semantic activity name. This is possible because initial activity models are used, i.e. generic but incomplete activity models provided by a domain expert. 

The clustering algorithm is divided into two steps, as introduced in Chapter \ref{cha:archi}: (i) a special pattern recognition algorithm that finds initial activity models' occurrences in the sensor activation dataset ($SA^3$), and (ii) a clustering step where object, sensor and action knowledge is used to define location, type and time based metrics which are used to aggregate actions to activities ($AA$). Following this division, the chapter is divided in three sections:

\begin{itemize}
 \item Section \ref{sec:clustering:sa3} describes the Semantic Activity Annotation algorithm ($SA^3$), which is presented as the initialization step of the clustering process.
 \item Section \ref{sec:clustering:ac} presents the second step of the clustering process, where actions are aggregated to corresponding initial activity clusters provided by $SA^3$.
 \item Section \ref{sec:clustering:sum} summarises this chapter and draws the extracted conclusions.
\end{itemize}
