
% Thesis Abstract -----------------------------------------------------


%\begin{abstractslong}    %uncommenting this line, gives a different abstract heading


\begin{abstracts}        %this creates the heading for the abstract page
\selectlanguage{british}
% Put your abstract or summary here.

Human activity recognition is one of the key competences for human adaptive technologies. The idea of such technologies is to adapt their services to human users, so being able to recognise what human users are doing is an important step to adapt services suitably. 

One of the most promising approaches for human activity recognition is the so-called knowledge-driven approach, which has already shown very interesting features and advantages. Knowledge-driven approaches allow using expert domain knowledge to describe activities and environments, providing efficient recognition systems. However, there are also some drawbacks, such as the usage of generic and static activity models, i.e. activities are defined by their generic features - they do not include personal specificities - and once activities have been defined, they do not evolve according to what users do.

This dissertation presents an approach to using data-driven techniques to evolve knowledge-driven activity models with a user's behavioural data. The approach includes a novel clustering process where initial incomplete models developed through knowledge engineering are used to detect activity clusters and aggregate new actions. Based on those activity clusters, a learning process is then designed to learn and model varying ways of performing activities in order to acquire complete and specialised activity models. The approach has been tested with real users' inputs, noisy sensors and demanding activity sequences. Results have shown that complete and specialised activity models are properly learned with success rates of 100\% at the expense of learning some false positive models.

\end{abstracts}

\begin{resumen}        %this creates the heading for the abstract page
\selectlanguage{spanish}
% Pon tu resumen aquí.

Resumen en espa\~nol aqu\'i


\end{resumen}

\begin{laburpena}        %this creates the heading for the abstract page
 \selectlanguage{basque} 
 % Idatzi hemen zure laburpena
 
 Laburpena euskaraz hemen
 
\end{laburpena}



%\end{abstractlongs}


% ---------------------------------------------------------------------- 
