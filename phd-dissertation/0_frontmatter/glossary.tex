% this file is called up by thesis.tex
% content in this file will be fed into the main document

% Glossary entries are defined with the command \nomenclature{1}{2}
% 1 = Entry name, e.g. abbreviation; 2 = Explanation
% You can place all explanations in this separate file or declare them in the middle of the text. Either way they will be collected in the glossary.

% required to print nomenclature name to page header
\markboth{\MakeUppercase{\nomname}}{\MakeUppercase{\nomname}}

% ----------------------- contents from here ------------------------
%

%
%
%% acronyms

\nomenclature{API}{Application Programming Interface}
\nomenclature{CHMM}{Coupled Hidden Markov Model}
\nomenclature{COM}{Continuous varied Order Multi Threshold}
\nomenclature{CRF}{Conditional Random Field}
\nomenclature{CSV}{Comma Separated Value}
\nomenclature{DBN}{Dynamic Bayesian Network}
\nomenclature{DL}{Description Logic}
\nomenclature{EC}{Event Calculus}
\nomenclature{GCP}{Google conditional Probabilities}
\nomenclature{GNU}{GNU's Not Unix}
\nomenclature{GPS}{Global Positioning System}
\nomenclature{HMM}{Hidden Markov Model}
\nomenclature{HTML}{Hyper Text Markup Language}
\nomenclature{HTTP}{Hyper Text Transfer Protocol}
\nomenclature{ID}{Identifier}
\nomenclature{IEEE}{Institute of Electrical and Electronics Engineers}
\nomenclature{ILSA}{Independent Lifestyle Assistant}
\nomenclature{IO}{Input/Output}
\nomenclature{JSON}{JavaScript Object Notation}
\nomenclature{KL}{Kullback-Leibler}
\nomenclature{LDS}{Linear Dynamical System}
\nomenclature{NN}{Nearest Neighbour}
\nomenclature{PEAT}{Planning and Execution Assistant and Trainer}
\nomenclature{RFID}{Radio Frequency Identification}
\nomenclature{SA³}{Semantic Activity Annotation Algorithm}
\nomenclature{SMC}{Sequential Monte Carlo}
\nomenclature{SVM}{Support Vector Machine}






