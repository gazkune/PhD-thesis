\section{Methodology}

A research strategy has been defined in order to achieve the statement presented
above. The strategy is defined as follows:

\begin{enumerate}
    \item Update knowledge by reviewing the literature in the area of remote
    laboratories, as well as attend conferences of this area.
    \item Design and develop different parts of the federation model,
    incrementally redefining them and adapting them to the new trends in the
    literature. 
    \item Attend conferences and workshops to present partial results and
    validate them with the existing state of the art.
    \item Experimentation and evaluation of the prototype at each particular
    stage.
    \item Contact experts at conferences, project meetings and consortia
    (the author is an active member of the GOLC --Global Online Laboratory
    Consortium). Contact for particular details by e-mail or by visiting other
    research centers (the author was a visiting researcher in the \emph{Center
    for Educational Computer Initiatives} in the Massachusetts Institute of
    Technology (\emph{MIT}) for 6 weeks in 2011 and in the \emph{Departamento de
    Ingeniería Eléctrica, Electrónica y de Control} of the Spanish Distance
    University (\emph{UNED}) in 2012 for 6 weeks, as well as a 2-week visit to
    the Labshare software development team in the University of Technology,
    Sydney).  
    \item Redesign with feedback from all this network, as well as the
    literature.
    \item Technical evaluation of the prototype.
    \item Dissemination of the results obtained during the research process.
\end{enumerate}

This methodology is illustrated in \myfig{fig:methodology}.
\InsertFig{Methodology}{fig:methodology}{Methodology used during the
dissertation}{}{0.9}{}
