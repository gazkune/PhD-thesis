\section{Hypothesis, Objectives and Scope}
\label{sec:intro:hypothesis}

%Write the hypothesis.

Based on the current state of activity modelling and recognition, the hypothesis of this dissertation is:

\vspace{0.5cm}

\noindent\fbox{
    \parbox{\textwidth}{
    \begin{hypo}
Using domain experts' previous knowledge as generic but incomplete activity models and data-driven learning techniques on user generated data, it is possible to learn accurately new actions to obtain personalised activity models for every user. 
\end{hypo}
}
}

\vspace{0.5cm}

To be able to validate this hypothesis the general goal of this dissertation is:

\vspace{0.5cm}

\noindent\fbox{
    \parbox{\textwidth}{
\begin{goal}
To design and implement a learning system that uses generic but incomplete activity models to analyse unlabelled activity sensor datasets and acquire personalised models which contain new actions.
\end{goal}
}
}

\vspace{0.5cm}


%Decompose hypothesis in smaller goals and objectives.

This general goal can be achieved by addressing the following more specific and measurable objectives:

\begin{enumerate}
 \item To study the current state of the art on knowledge- and data-driven activity modelling and recognition.
 \item To choose a knowledge representation formalism and design proper structures to represent domain experts' knowledge.
 \item To design and implement a multiple step learning algorithm which uses previous knowledge and user generated data to obtain personalised activity models.
 \item To identify the evaluation methodology for the learning system which better addresses the requirements of the system.
 \item To validate the obtained results quantitatively, with the objective of capturing the 100\% of real activity models performed by a user. 
\end{enumerate}

The resulting learning system should also fulfil the following requirements:

\begin{enumerate}
 \item User independence: the learning system should be able to acquire personalised models for any user.
 \item Use the same generic activity models for every user: to show that incomplete activity models provided by experts are really generic, the same models have to be used for every user.
 \item Environment independence: the knowledge representation formalism and the learning algorithm should be defined to cope with different environments. As such, representing new locations and objects of a different environment should be straightforward.
\end{enumerate}

%Describe constraints and limitations.

The work presented in this dissertation does not deal with the following conditions:

\begin{enumerate}
 \item Unknown activities: this dissertation does not propose any method to identify and learn unknown activities. Personalised models of already defined activities are learnt, thus leaving the integration of other techniques to learn unknown activities for future work.
 \item Multiple users being monitored: it is assumed that only one user will be monitored in a particular experiment and dataset. Considering multi-user monitoring approaches is beyond the scope of this research work.
 \item Concurrent and interwoven activities: users are not allowed to perform activities concurrently, i.e. once they start an activity, they have to finish it before starting another activity. Concurrent and interwoven activities pose challenges that will not be addressed in this dissertation.
 %\item Online learning: the learning system will not be able to work online. Datasets will be processed offline using batch training systems.
\end{enumerate}



