\section{Thesis Outline}
\label{sec:intro:outline}

The thesis is structured in seven chapters.

Chapter \ref{cha:introduction} is the current chapter. It presents the context and motivation of the research, as well as the hypothesis, goals and scope. To achieve those goals and validate the hypothesis, a research methodology is proposed. Finally, the contributions of the dissertation and their location in the document are also shown.

Chapter \ref{cha:soa} describes the state of the art relevant to the dissertation. The most important activity monitoring, modelling and recognition approaches are presented providing a well structured taxonomy. The chapter also shows the position in such taxonomy of the research performed in this dissertation.

Chapter \ref{cha:archi} establishes the basis of the dissertation, giving a high-level view of the whole system. First of all, the chapter describes in detail ontology-based activity modelling approaches, which serve as reference for the work carried out in this dissertation. Afterwards, a set of definitions and constraints is posed to establish the scope of the research clearly. The knowledge representation formalism and structures are presented and the input data of the system is identified. Finally, the intuition behind the learning system and a high-level design are described.

Chapter \ref{cha:clustering} describes the proposed activity clustering algorithm, which is divided into two main steps: initialisation and action aggregation. Both steps are described in detail, providing pseudo-code for their implementation.

Chapter \ref{cha:learner} explains the last stage of the learning process. Design decisions are explained and a rigorous analysis of all available information to learn activity models is done. The learning algorithm's steps are described in detail and pseudo-code for its implementation is provided.

Chapter \ref{cha:evaluation} evaluates the complete learning system and its constituent parts. For this purpose, the most relevant evaluation methodologies for activity recognition systems are shown and their suitability to evaluate the learning system is analysed. The chapter proposes a new hybrid evaluation methodology, based on surveys to users and simulation tools. Afterwards, evaluation scenarios and metrics are presented and obtained results are shown, leading to a discussion of the learning system, its advantages and drawbacks.

Finally, Chapter \ref{cha:conclusions} summarises the dissertation, draws the conclusions of this research work and shows the related future lines of work.