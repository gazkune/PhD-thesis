
% this file is called up by thesis.tex
% content in this file will be fed into the main document

%: ----------------------- introduction file header -----------------------


\begin{savequote}[50mm]
The first step, my son, which one makes in the world, is the one on which depends the rest of our days.
\qauthor{François Marie Arouet, Voltaire}
\end{savequote}

\chapter{Introduction}
\label{cha:introduction}

% the code below specifies where the figures are stored
\ifpdf
    \graphicspath{{1_introduction/figures/PDF/}{1_introduction/figures/PNG/}{1_introduction/figures/}}
\else
    \graphicspath{{1_introduction/figures/EPS/}{1_introduction/figures/}}
\fi

\sloppy \letra{E}{xplain} briefly the concept of activity recognition and its relevance, citing some important papers (1 paragraph). 

Describe how activity recognition can be done using sensors to monitor human actions and the two main currents in activity recognition: data-driven and knowledge-driven (1 paragraph).

Describe briefly the main contributions of this dissertation (1 paragraph).

Describe the structure of the chapter.

%The remainder of this chapter is structured as follows: \mysec{sec:intro:context} explains the context of the research. \mysec{sec:intro:statement} presents the core statement of this dissertation, and finally \mysec{sec:intro:contributions} summarizes the rest of the contributions of this dissertation.

Describe the structure of the whole dissertation.

%The dissertation is structured as follows: \mycha{cha:sota} describes the state of the art relevant to the dissertation, \mycha{cha:weblab-deusto} describes the WebLab-Deusto system, on which the federation model has been implemented, \mycha{cha:federation-model} describes the proposed federation model, and \mycha{cha:evaluation} evaluates both the federation model and its implementation in WebLab-Deusto. Finally, \mycha{cha:conclusions} summarizes the dissertation and shows the related future lines of work.

