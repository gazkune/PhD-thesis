
% this file is called up by thesis.tex
% content in this file will be fed into the main document

%: ----------------------- introduction file header -----------------------


\begin{savequote}[50mm]
The first step, my son, which one makes in the world, is the one on which depends the rest of our days.
\qauthor{François Marie Arouet, Voltaire}
\end{savequote}

\chapter{Introduction}
\label{cha:introduction}

% the code below specifies where the figures are stored
\ifpdf
    \graphicspath{{1_introduction/figures/PDF/}{1_introduction/figures/PNG/}{1_introduction/figures/}}
\else
    \graphicspath{{1_introduction/figures/EPS/}{1_introduction/figures/}}
\fi

%Explain briefly the concept of activity recognition and its relevance, citing some important papers (1 paragraph). 

\sloppy \letra{H}{uman} activity recognition aims at recognising what a human is doing in different environments and domains. To perform activity recognition, different kinds of sensors have to be deployed in human-populated environments to monitor inhabitants' behaviours and capture environmental changes generated by human actions. The information provided by those sensors has to be processed through data analysis techniques and/or knowledge representation formalisms to create appropriate activity models and subsequently use them for activity recognition. Thus using sensing and computing capabilities, activities performed by humans can be detected and recognised.

Being able to recognise what human beings are doing in their daily life can open the door to a lot of possibilities in diverse areas. Technology is becoming more and more human-centred in order to provide personalised services. In other words, technological services have to adapt to human users and not the other way around.

Following that trend, human activity recognition becomes a natural enabler to adaptive technologies. As such, it has become an important research topic in areas such as pervasive and mobile computing \citep{Choudhury2008}, ambient assisted living \citep{Philipose2004}, social robotics \citep{Fong2003a}, surveillance-based security \citep{Fernandez-Caballero2012} and context-aware computing \citep{Laerhoven2001}. 

%Describe how activity recognition can be done using sensors to monitor human actions and the two main currents in activity recognition: data-driven and knowledge-driven (1 paragraph).

The scientific community has developed two main approaches to solve activity recognition, namely the data-driven and knowledge-driven approaches. Data-driven approaches make use of large-scale datasets of sensors to learn activity models using data mining and machine learning techniques. On the other hand, knowledge-driven approaches exploit rich prior knowledge in the domain of interest to build activity models using knowledge engineering and management technologies.

%Describe briefly the main contributions of this dissertation (1 paragraph).

For knowledge-driven activity recognition systems, a widely recognised drawback is that activity models are usually static, i.e. once they have been defined, they cannot be automatically adapted to users' specificities \citep{Chen2014}. This is a very restrictive limitation, because it is not generally possible to define complete activity models for every user. Domain experts have the necessary knowledge about activities, but this knowledge may not be enough to generate complete models in all the cases. To make knowledge-driven activity recognition systems work in real world applications, activity models have to evolve automatically to adapt to users' varying behaviours. It turns out that model adaptability and evolution are aspects that can be properly addressed by data-driven approaches. Hence, the objective of this dissertation is to use data-driven techniques to make knowledge-driven activity models evolve automatically based on sensor data generated by specific users.

%Describe the structure of the chapter.

The remainder of this chapter is structured as follows: Section \ref{sec:intro:context} explains the context and motivation of the research. Afterwards, Section \ref{sec:intro:hypothesis} formulates the hypothesis, derived goals and the scope of the work. Section \ref{sec:intro:methodology} describes the followed research methodology to achieve the goals, and \mysec{sec:intro:contributions} summarises the scientific and technical contributions of this dissertation. The chapter finalises with an outline of the dissertation in Section \ref{sec:intro:outline}.

%Describe the structure of the whole dissertation.

%The rest of the dissertation is structured as follows: Chapter \ref{cha:soa} describes the state of the art relevant to the dissertation, Chapter \ref{cha:archi} establishes the basis of the dissertation, giving a high-level view of the whole system. Chapter \ref{cha:clustering} describes the proposed activity clustering algorithm, and Chapter \ref{cha:learner} explains how the learning process has been developed. Chapter \ref{cha:evaluation} evaluates the complete learning system and its constituent parts. Finally, Chapter \ref{cha:conclusions} summarises the dissertation and shows the related future lines of work.

