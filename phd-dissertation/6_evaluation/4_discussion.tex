\section{Summary and Conclusions}
\label{sec:evaluation:conclusions}

After analysing the most used evaluation methodology in the community of activity recognition (Section \ref{sec:evaluation:methodology}), it has been concluded that using such methodology to properly assess the performance of the EAM learning system is not feasible. In consequence a novel evaluation methodology has been described in this chapter (Section \ref{subsec:evaluation:hybrid}), following the trend of several researchers who have already analysed the benefits of using simulated environments for activity recognition. Combining a synthetic dataset generator tool with surveys to real users, realistic experimental set-ups can be prepared, introducing varying time lapses and sensor noise. One of the most important novelties of the presented evaluation methodology is to provide a tool to model human behaviour reliably using specially designed surveys. 

To test all the relevant parts of the EAM learning system, several experiments or evaluation scenarios have been prepared in Section \ref{sec:evaluation:scenarios}. Those experiments have been prepared using the hybrid evaluation methodology. The performance of every algorithm has been measured based on well defined metrics.

The results obtained for all the experiments for $SA^3$, the complete clustering process and for the EAM learning system can be considered very good, even having high levels of sensor noise - remember that the complete scenario contains more than 38\% more sensor activations due to sensor noise than the ideal scenario -. For instance, experiments for $SA^3$ show how accurate is the algorithm when detecting time locations of performed activities. It cannot label correctly every action of a dataset, but it can very accurately distinguish between action sequences describing different activities. This initialisation, which is specially robust to sensor positive noise, is the key step to make sure that all action sequences will be properly captured in the end of the clustering process. The obtained results show that the $AA$ algorithm analyses every action suitably after the initialisation step, using all the knowledge represented in the context knowledge file. Even though false positives can be generated at this step, the vast majority of actions are properly tagged, taking advantage of action type, location and the defined three time metrics. This can be seen in the high precision and recall values obtained in both scenarios. For example, precision is above 97\% in the complete scenario using the third time approach, whereas recall is above 98\% (see Table \ref{tab-r-comparative-complete}). 

The combined results of both algorithms, $SA^3$ and $AA$, allow capturing effectively all the action sequences which have been executed by a user for a concrete activity. At this stage, the most important thing is that the action clusters obtained for each activity, actually contain the real action sequences executed by users. However, there are also spurious variations of those real sequences due to sensor noise and clustering errors. Based on those action clusters $AML$ implements a similarity-based outlier detection algorithm. Such algorithm encodes action sequence information using the Jaccard coefficient and limits the problem of finding spurious models to detect outliers in the similarity space. Using conservative approaches to avoid removing any real action sequence, a statistical outlier detection approach is run. The approach works well because spurious action sequences are very similar to real action sequences. More concretely, as spurious action sequences are slight variations of real action sequences, their similarity values are higher than expected for the similarity distribution of a given activity. $AML$ establishes a threshold using statistics that are robust to outliers. This threshold separates those similarity values that can be considered normal from anomalous or outlier similarity values. The result is that all real activity models are correctly learned, while the number of spurious or false models is acceptable. 

