
% this file is called up by thesis.tex
% content in this file will be fed into the main document

%: ----------------------- introduction file header -----------------------
\begin{savequote}[50mm]
Science is a way of trying not to fool yourself. The first principle is that you must not fool yourself, and you are the easiest person to fool.
\qauthor{Richard Feynman}
\end{savequote}


\chapter{Evaluation}
\label{cha:evaluation}

% the code below specifies where the figures are stored
\ifpdf
    \graphicspath{{6_evaluation/figures/PDF/}{6_evaluation/figures/PNG/}{6_evaluation/figures/}}
\else
    \graphicspath{{6_evaluation/figures/EPS/}{6_evaluation/figures/}}
\fi

\letra{T}{his} chapter describes and analyses the proposed evaluation methodology for the extended activity model learning system described throughout Chapters \ref{cha:archi}, \ref{cha:clustering} and \ref{cha:learner}. To test the learning system properly, detailed data from various users is needed, containing different ways of performing the same activity by the same user. In order to achieve such datasets, the proposed evaluation methodology is based on surveys to users and a synthetic dataset generator tool. Surveys allow capturing how users perform activities in terms of actions, while the synthetic dataset generator uses survey information to generate sensor activation datasets, introducing different kinds of sensor noise. 

Once the methodology is described and discussed, evaluation scenarios and metrics will be introduced. To assess the performance of the learning system in various situations, several evaluation scenarios have been prepared, considering different kinds of sensor noise and set-ups. Similarly, and based on the available literature, the most significant performance metrics have been chosen to measure the performance of the approach.

Applying the evaluation methodology on the prepared scenarios and selected metrics, results are obtained. Those results are widely discussed in this chapter and compared to the objectives defined in Chapter \ref{cha:introduction}.

The chapter is divided in three sections: Section \ref{sec:evaluation:methodology} introduces the typical evaluation methodology used for activity recognition, detects its drawbacks and proposes a new evaluation methodology which is more appropriate to validate the learning system. Section \ref{sec:evaluation:scenarios} presents evaluation scenarios, metrics, results and discussions divided in three main parts: $SA³$ performance (Section \ref{subsec:evaluation:sa3}), the complete activity clustering performance (Section \ref{subsec:evaluation:clustering}) and finally the EAM learning system performance (Section \ref{subsec:evaluation:eam}). All three sections discuss the results and compare them with the objectives identified in the beginning of this dissertation (Chapter \ref{cha:introduction}). To conclude, Section \ref{sec:evaluation:conclusions} provides a global view of the system performance, summarises the contents of the chapter and presents the most relevant conclusions.