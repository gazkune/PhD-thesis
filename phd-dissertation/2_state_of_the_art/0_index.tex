
% this file is called up by thesis.tex
% content in this file will be fed into the main document

%: ----------------------- introduction file header -----------------------
\begin{savequote}[50mm]
If I have seen further it is by standing on the shoulders of giants
\qauthor{Isaac Newton}
\end{savequote}

\chapter{State of the Art}
\label{cha:soa}

% the code below specifies where the figures are stored
\ifpdf
    \graphicspath{{2_state_of_the_art/figures/PDF/}{2_state_of_the_art/figures/PNG/}{2_state_of_the_art/figures/}}
\else
    \graphicspath{{2_state_of_the_art/figures/EPS/}{2_state_of_the_art/figures/}}
\fi

%\letra{D}{escribe} how the state of the art has been structured in this chapter, referencing sections and commenting on their contents. Justify the topics covered by the chapter.

\letra{H}{uman} activity recognition is a broad research area which can be analysed from many different points of view. The objective of this chapter is to provide a high-level picture of the current status of the research topic, to then describe more in detail the concrete approaches that are related to the work presented in this dissertation. 

First of all, Section \ref{sec:soa:har} introduces the concept of \textit{Human Activity Recognition}, shows its main applications and provides a high-level taxonomy of activity recognition approaches. Section \ref{sec:soa:sensor} describes in detail the category in which this dissertation fits: \textit{Sensor-Based Activity Recognition}. Based on the sensors used for activity monitoring, Section \ref{sec:soa:monitoring} presents a more detailed taxonomy for the category of sensor-based activity recognition. On the other hand, the following sections analyse the main currents of activity modelling: Section \ref{sec:soa:datadriven} presents \textit{Data-Driven Approaches} , while Section \ref{sec:soa:knowledgedriven} describes in detail \textit{Knowledge-Driven Approaches}. A summary and comparison of both activity modelling currents is provided in Table \ref{tab:soa:comparison}. Some recent work has settled the basis to combine both currents, giving as result the \textit{Hybrid approaches}, as shown in Section \ref{sec:soa:hybrid}. This dissertation is another step in order to achieve hybrid approaches that can combine effectively the best features of knowledge- and data-driven approaches. The chapter finalises with Section \ref{sec:soa:summary}, where a concise summary is provided alongside with some conclusions.