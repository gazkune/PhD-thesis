\section{Hybrid Approaches}
\label{sec:soa:hybrid}

%Describe here the approach Chen approach to learn non-modelled activities and descriptive properties for modelled activities. Any other reference for hybrid approaches?

Analysing carefully the advantages and disadvantages of knowledge- and data-driven approaches, it can be seen that they are complementary at some extent. For instance, while knowledge-driven approaches are weak in handling uncertainty and temporal information, data-driven ones are good at it. On the other hand, while data-driven approaches cannot build reusable generic activity models, knowledge-driven approaches can. This complementarity has recently raised the need to research on \textit{hybrid approaches to activity modelling}. The main idea of hybrid approaches is to fuse data- and knowledge-driven approaches in order to solve the problems of both approaches in a single system.

Tran et al. \cite{Tran2008} and Healaoui et al. \cite{Helaoui2011a} use Markov Logic Networks (MLN) to model temporal relations between actions and activities. The first one uses MLNs to naturally integrate common sense reasoning with uncertain analyses produced by computer vision algorithms for object detection, tracking and movement recognition. The second work is focused on providing activity models for interleaved and concurrent executions using qualitative and quantitative temporal relationships. With a similar objective, Steinhauer et al. \cite{Steinhauer2010} combine HMMs with Allen logic, providing another hybrid approach example. All those approaches encode and use temporal knowledge and rely on automatically extracting these temporal patterns from datasets.

Another hybrid approach example is provided by Riboni et al. \cite{Riboni2011a}. Their approach combines statistical inference techniques with ontological activity modelling and recognition. Statistical inferencing is performed based on raw data retrieved from body-worn sensors (e.g., accelerometers) to predict the most probable activities. Then, symbolic reasoning is applied to refine the results of statistical inferencing by selecting the set of possible activities performed by a user based on his/her current context. By decoupling the use of context information, statistical inferencing becomes more manageable in terms of necessary training data, while symbolic reasoning can more effectively select candidate activities taking into account context-dependent ontological relationships.

Finally, Chen et al. \cite{Chen2014} present an ontology-based hybrid approach to activity modelling. They combine knowledge-based activities with specially developed learning algorithms to overcome the ``cold start'' problem, model reusability and incompleteness. The approach uses semantic technologies as a conceptual backbone and technology enablers for ADL modelling, classification and learning. The distinguishable feature of the approach from existing approaches is that ADL modelling is not a one-off effort, instead, a multi-phase iterative process that interleaves knowledge-based model specifications and data-driven model learning. The process consists of three key phases. In the first phase the initial seed ADL models are created through ontological engineering by leveraging domain knowledge and heuristics, thus solving the ``cold start'' problem. Ontological activity modelling creates activity models at two levels of abstractions, namely as ontological activity concepts and their instances respectively. Ontological activity concepts represent generic coarse-grained activity models applicable and reusable for all users, thus solving the reusability problem. The seed ADL models are then used in applications for activity recognition at the second phase. In the third phase, the activity classification results from the second phase are analysed to discover new activities and user profiles. These learnt activity patterns are in turn used to update the ADL models, thus solving the incompleteness problem. Once the first phase completes, the remaining two-phase process can be iterated many times to incrementally evolve the ADL models, leading to complete, accurate and up-to-date ADL models. In consequence, this approach is a first step to solve the static nature of activity models in knowledge-driven systems. However, the approach has a limitation: it considers that seed activity models contain all the actions performed by any user to perform an activity and hence, it is limited to learn only descriptive properties. This means that if different users perform the same activity by means of different sets of actions, the system will not be able to learn those distinct actions. 