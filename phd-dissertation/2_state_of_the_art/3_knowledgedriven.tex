\section{Knowledge-Driven Approaches}
\label{sec:soa:knowledgedriven}

Knowledge-driven activity modelling is motivated by real world observations that for most activities of daily living and working, the list of objects required for a particular activity is limited and functionally similar. Even if the activity can be performed in different ways the number and type of involved objects do not vary significantly. For example, it is common sense that the activity “make coffee” consists of a sequence of actions involving a coffee pot, hot water, a cup, coffee, sugar and milk; the activity “brush teeth” contains actions involving a toothbrush, toothpaste, water tap, cup and towel. On the other hand, as humans have different life styles, habits or abilities, they may perform various activities in different ways. For instance, one may like strong white coffee and another may prefer a special brand of coffee. Even for the same type of activity (e.g., making white coffee), different individuals may use different items (e.g., skimmed milk or whole milk) and in different orders (e.g., adding milk first and then sugar, or vice versa). Such domain-dependent activity-specific prior knowledge provides valuable insights into how activities can be constructed in general and how they can be performed by individuals in specific situations.

Similarly, knowledge-driven activity recognition is founded upon the observations that most activities, in particular, routine activities of daily living and working, take place in a relatively specific circumstance of time, location and space. The space is usually populated with events and entities pertaining to the activities, forming a specific environment for specific purposes. For example, brushing teeth is normally undertaken twice a day in a bathroom in the morning and before going to bed and involves the use of toothpaste and a toothbrush; meals are made in a kitchen with a cooker roughly three times a day. The implicit relationships between activities, related temporal and spatial context and the entities involved (objects and people) provide a diversity of hints and heuristics for inferring activities.

Knowledge-driven activity modelling and recognition intends to make use of rich domain knowledge and heuristics for activity modelling and pattern recognition. The rationale is to use various methods, in particular, knowledge engineering methodologies and techniques, to acquire domain knowledge. The captured knowledge can then be encoded in various reusable knowledge structures, including activity models for holding heuristics and prior knowledge in performing activities, and context models for holding relationships between activities, objects and temporal and spatial contexts. Comparing to data-driven activity modelling that learns models from large-scale datasets and recognises activities through data intensive processing methods, knowledge-driven activity modelling avoids a number of problems, including the requirement for large amounts of observation data, the inflexibility that arises when each activity model needs to be computationally learned, and the lack of reusability that results when one person’s activity model is different from another’s. 

Knowledge structures can be modelled and represented in different forms, such as schemas, rules or networks. This will decide the way and the extent to which knowledge is used for following processing such as activity recognition, prediction and assistance. In terms of the manner in which domain knowledge is captured, represented and used, knowledge-driven approaches to activity modelling and recognition can be roughly classified into three main categories as presented in the following sections.

\subsection{Mining-based approach}

The objective of a mining-based approach is to create activity models by mining existing activity knowledge from publicly available sources. More specifically, given a set of activities, the approach seeks to discover from the text corpuses a set of objects used for each activity and extract object usage information to derive their associated usage probabilities. The approach essentially views activity model as a probabilistic translation between activity names (e.g., “make coffee”) and the names of involved objects (e.g., “mug”, “milk”). As the correlations between activities and their objects are common-sense prior knowledge (e.g., most of us know how to carry out daily activities), such domain knowledge can be gleaned in various sources such as how-tos (e.g., those at \textit{ehow.com} or \textit{wikihow.com}), recipes (e.g., from \textit{epicurious.com}), training manuals, experimental protocols, and facility/device user manuals.

A mining-based approach consists of a sequence of distinct tasks. Firstly it needs to identify activities of concern and relevant sources that describe these activities. Secondly, it uses various methods, predominantly information retrieval and analysis techniques, to retrieve activity definitions from specific sources and extract phrases that describe the objects used during the performance of the activity. Then algorithms, predominantly probabilistic and statistic analysis methods such as co-occurrences and association are used to estimate the object-usage probabilities. Finally, the mined object and usage information is used to create activity models such as a HMM that can be used further for activity recognition. 

Mining-based activity modelling was initially investigated by researchers from Intel Research \cite{Wu2007} \cite{Wyatt2005}. Perkowitz et al. \cite{Perkowitz2004} proposed the idea of mining the Web for large-scale activity modelling. They used the QTag tagger to tag each word in a sentence with its part of speech (POS) and a customized regular expression extractor to extract objects used in an activity. They then used the Google Conditional Probabilities (GCP) APIs to determine automatically the probability values of object usage. The mined object and their usage information are then used to construct DBN models through Sequential Monte Carlo (SMC) approximation. They mined the website \textit{ehow.com} for roughly 2300 directions on performing domestic tasks (from “boiling water in the microwave” to “change your air filter”), and the website \textit{ffts.com} and \textit{epicurious.com} for a further 400 and 18,600 recipes respectively, generating a total 21,300 activity models. Using the DBN activity models they have performed activity recognition for a combination of real user data and synthetic data. While initial evaluation results were positive, the drawback was that there are no mechanisms to guarantee the mined models capturing completely the sequence probabilities and the idiosyncrasy of certain activities. The inability to capture such intrinsic characteristics may limit the model's accuracy in real deployments.

Wyatt et al. \cite{Wyatt2005} followed Perkowitz’s approach by mining the Web to create DBN activity models. However, this group extended the work in three aspects, aiming to address the idiosyncrasies and to improve model accuracy. To cover the wide variety of activity definition sources, they mined the Web in a more discriminative way in a wider scope. They did this by building a specialised genre classifier trained and tested with a large number of labelled Web pages. To enhance model applicability, they used the mined models as base activity models and then exploited the Viterbi Algorithm and Maximum Likelihood to learn customized activity parameters from unsegmented, unlabelled sensor data. In a bid to improve activity recognition accuracy they also presented a bootstrap method that produced labelled segmentations automatically. Then they used the Kullback-Leibler (KL) divergence to compute activity similarity.

A difficulty in connecting mined activities with tagged objects \cite{Perkowitz2004} \cite{Wyatt2005} is that the activity models may refer to objects synonymously. For example, both a “mug” and “cup” can be used for make tea; both a “skillet” and “frying pan” may be used for making pasta. This leads to a situation that one activity may have different models with each having the same activity name but different object terms. To address this, Tapia et al. \cite{Tapia2006} proposed to extract collections of synonymous words for the functionally-similar objects automatically from WordNet, an online lexical reference system for the English language. The set of terms for similar objects is structured and represented in a hierarchical form known as the object ontology. With the similarity measure provided by the ontology, an activity model will not only cover a fixed number of object terms but also any other object terms that are in the same class in the ontology. 

Another shortcoming of early work in the area \cite{Perkowitz2004} \cite{Wyatt2005} is that the segmentation is carried out in sequential order based on the duration of an activity. As the duration of performing a specific activity may vary substantially from one to another, this may give rise to applicability issues. In addition, in sequential segmentation one error in one segment may affect the segmentations of the subsequent traces. To tackle this, Palmes et al. \cite{Palmes2010} proposed an alternate method for activity segmentation and recognition. Instead of relying on the order of object use, they exploited the discriminative trait of the usage frequency of objects in different activities. They constructed activity models by mining the Web and extracting relevant objects based on their weights. The weights are then utilised to recognise and segment an activity trace containing a sequence of objects used in a number of consecutive and non-interleaving activities. To do this, they proposed an activity recognition algorithm, KeyExtract, which uses the list of discriminatory key objects from all activities to identify the activities present in a trace. They further proposed two heuristic segmentation algorithms, MaxGap and MaxGain, to detect the boundary between each pair of activities identified by KeyExtract. Boundary detection is based on the calculation, aggregation, and comparison of the relative weights of all objects sandwiched in any two key objects representing adjacent activities in a trace. Though the mining based approach has a number of challenges relating to information retrieval, relation identification and the disambiguation of term meaning, nevertheless, it provides a feasible alternative to model large amount of activities. Initial research has demonstrated the approach is promising. 

Mining-based approaches are similar to data-driven approaches in that they all adopt probabilistic or statistical activity modelling and recognition. But they are different from each other in the way the parameters of the activity models are decided. The mining-based approaches make use of publicly available data sources avoiding the “cold start” problem. Nevertheless they are weak in dealing with idiosyncrasies of activities. On other hand, data-driven approaches have the strength of generating personalised activity models, but they suffer from issues such as “cold start” and model reusability for different users.

\subsection{Logic-based approaches}

A logic-based approach views an activity as a knowledge model that can be formally specified using various logical formalisms. From this perspective, activity modelling is equivalent to knowledge modelling and representation. As such, systematic knowledge engineering methodologies and techniques are used for domain knowledge acquisition and formal construction of activity structures. Knowledge representation formalisms or languages are used to represent these knowledge models and concrete knowledge instances, thus enabling inference and reasoning. In this way, activity recognition and other advanced application features such as prediction and explanation can be mapped to knowledge-based inference such as deduction, induction and abduction. 

Logic-based approaches are composed of a number of distinct tasks. Even though each task can be undertaken in different ways the role of each task is specific and unique. Normally the first step is to carry out knowledge acquisition, which involves eliciting knowledge from various knowledge sources such as domain experts and activity manuals. The second step is to use various knowledge modelling techniques and tools to build reusable activity structures. This will be followed by a domain formalization process in which all entities, events and temporal and spatial states pertaining to activities, along with axioms and rules, are formally specified and represented using representation formalism. This process usually generates the domain theory. The following step will be the development of a reasoning engine in terms of knowledge representation formalisms to support inference. In addition, a number of supportive system components will be developed, which are responsible for aggregating and transforming sensor data into logical terms and formula. With all functional components in place, activity recognition proceeds by passing the logical representation of sensor data onto the reasoning engine. The engine performs logical reasoning, e.g., deduction, abduction or induction, against the domain theory. The reasoning will extract a minimal set of covering models of interpretation from the activity models based on a set of observed actions, which could semantically explain the observations.

There exist a number of logical modelling methods and reasoning algorithms in terms of logical theories and representation formalisms. One thread of work is to map activity recognition to the plan recognition problem in the well studied artificial intelligence field \cite{Carberry2001}. The problem of plan recognition can be stated in simple terms as: given a sequence of actions performed by an actor, how to infer the goal pursued by the actor and also to organise the action sequence in terms of a plan structure. Kautz et al. \cite{Kautz1991} adopted first-order axioms to build a library of hierarchical plans. They proposed a set of hypotheses such as exhaustiveness, disjointedness and minimum cardinality to extract a minimal covering model of interpretation from the hierarchy, based on a set of observed actions. Wobke \cite{Wobcke2002} extends Kautz’s work using situation theory to address the different probabilities of inferred plans by defining a partial order relation between plans in terms of levels of plausibility. Bouchard et al. \cite{Bouchard2006} borrow the idea of plan recognition and apply it to activity recognition. They use action Description Logic (DL) to formalize actions and entities and variable states in a smart home to create a domain theory. They model a plan as a sequence of actions and represent it as a lattice structure, which, together with the domain theory, provides an interpretation model for activity recognition. As such, given a sequence of action observations, activity recognition amounts to reasoning against the interpretation model to classify the actions through a lattice structure. It was claimed that the proposed DL models can organize the result of the recognition process into a structured interpretation model in the form of lattice, rather than a simple disjunction of possible plans without any classification. This minimises the uncertainty related to the observed actor’s activity by bounding the plausible plans set.

Another thread of work is to adopt the Event Calculus (EC) \cite{Shanahan1997} formalism, a highly developed logical theory of actions, for activity recognition and assistance. The EC formalizes a domain using fluents, events and predicates. Fluents are any properties of the domain that can change over time. Events are the fundamental instrument of change. All changes to a domain are the result of named events. Predicates define relations between events and fluents that specify what happens when and which fluents hold at what times. Predicates also describe the initial situation and the effects of events. Chen et al. \cite{Chen2008} proposed an EC-based framework in which sensor activations are modelled as events, and object states as properties. In addition, they developed a set of high- level logical constructors to model compound activities, i.e. the activities consisting of a number of sequential and/or parallel events. In the framework, an activity trace is simply a sequence of events that happen at different time points. Activity recognition is mapped to deductive reasoning tasks, e.g., temporal projection or explanation, and activity assistance or hazard prevention is mapped to abductive reasoning tasks. The major strength of this work is its capability to address temporal reasoning and the use of compound events to handle uncertainty and flexibility of activity modelling. 

Logic-based approaches are totally different from data-driven approaches in the way activities are modelled and the mechanisms activities are recognised. They do not require pre-existing large-scale datasets, and activity modelling and recognition is semantically clear and elegant in computational reasoning. It is easy to incorporate domain knowledge and heuristics for activity models and data fusion. The weakness of logical approaches is their inability or inherent infeasibility to represent fuzziness and uncertainty even though there are recent works trying to integrate fuzzy logics into the logical approaches. Another drawback is that logical activity models are viewed as one-size-fits-all, inflexible for adaptation to different users’ activity habits.

\subsection{Ontology-based approaches}
\label{subsec:soa:ontology}

Using ontologies for activity recognition is a recent endeavour and has gained growing interest. In the vision-based activity recognition community, researchers have realised that symbolic activity definitions based on manual specification of a set of rules suffer from limitations in their applicability, because the definitions are only deployable to the scenarios for which they have been designed. There is a need for a commonly agreed explicit representation of activity definitions or an ontology. Such ontological activity models are independent of algorithmic choices, thus facilitating portability, interoperability and reuse and sharing of both underlying technologies and systems. Chen et al. \cite{Chen2004} propose activity ontologies for analysing social interaction in nursing homes, Hakeem et al. \cite{Hakeem2004} for the classification of meeting videos, and Georis et al. \cite{Georis2004} for activities in a bank monitoring setting. To consolidate these efforts and to build a common knowledge base of domain ontologies, a collaborative effort has been made to define ontologies for six major domains of video surveillance. This has led to a video event ontology \cite{Nevatia2004} and the corresponding representation language \cite{Francois2005}. For instance, Akdemir \cite{Akdemir2008} used the video event ontologies for activity recognition in both bank and car park monitoring scenarios. In principle these studies use ontologies to provide common terms as building primitives for activity definitions. Activity recognition is performed using individually preferred algorithms, such as rule-based systems \cite{Hakeem2004} and finite-state machines \cite{Akdemir2008}. 

In the dense sensing-based activity recognition community, ontologies have been utilised to construct reliable activity models. Such models are able to match different object names with a term in an ontology which is related to a particular activity. For example, a Mug sensor event could be substituted by a Cup event in the activity model MakeTea as Mug and Cup can both be used for the MakeTea activity. This is particularly useful to address model incompleteness and multiple representations of terms. Tapia et al. \cite{Tapia2006} generate a large object ontology based on functional similarity between objects from WordNet, which can complete mined activity models from the Web with similar objects. Yamada et al. \cite{Yamada2007} use ontologies to represent objects in an activity space. By exploiting semantic relationships between things, the reported approach can automatically detect possible activities even given a variety of object characteristics including multiple representation and variability. Similar to vision-based activity recognition, these studies mainly use ontologies to provide activity descriptors for activity definitions. Activity recognition can then be performed based on probabilistic and/or statistical reasoning \cite{Tapia2006} \cite{Yamada2007}.

Ontology-based modelling and representation has been applied to general ambient assisted living. Latfi et al. \cite{Latfi2007} propose an ontological architecture of a telehealth-based smart home aiming at high-level intelligent applications for elderly persons suffering from loss of cognitive autonomy. Klein et al. \cite{Klein2007} developed an ontology-centred design approach to create a reliable and scalable ambient middleware. Chen et al. \cite{Chen2009} pioneered the notion of semantic smart homes in an attempt to leverage the full potential of semantic technologies in the entire lifecycle of assistive living i.e. from data modelling, content generation, activity representation, processing techniques and technologies to assist with the provision and deployment. While these endeavours, together with existing work in both vision- and dense sensing-based activity recognition, provide solid technical underpinnings for ontological data, object, sensor modelling and representation, there is a gap between semantic descriptions of events/objects related to activities and semantic reasoning for activity recognition. Most works use ontologies either as mapping mechanisms for multiple terms of an object \cite{Tapia2006} or the categorization of terms \cite{Yamada2007} or a common conceptual template for data integration, interoperability and reuse \cite{Latfi2007} \cite{Klein2007} \cite{Chen2009}. Activity ontologies which provide an explicit conceptualization of activities and their interrelationships have only recently emerged and have been used for activity recognition. Chen et al. \cite{Chen2009b} \cite{Chen2012a} proposed and developed an ontology-based approach to activity recognition. They constructed context and activity ontologies for explicit domain modelling. Sensor activations over a period of time are mapped to individual contextual information and then fused to build a context at any specific time point. They made use of subsumption reasoning to classify the constructed context based on the activity ontologies, thus inferring the ongoing activity. Ye et al. \cite{Ye2011} developed an upper activity ontologies that facilitates the capture of domain knowledge to link the meaning implicit in elementary information to higher-level information that is of interest to applications. Riboni et al. \cite{Riboni2011b} investigated the use of activity ontologies, in particular, the new feature of rule representation and rule-based reasoning from OWL2, to model, represent and reason complex activities.

Compared with data-driven and mining-based approaches, ontology-based approaches offer several compelling features: firstly, ontological ADL models can capture and encode rich domain knowledge and heuristics in a machine understandable and processable way. This enables knowledge based intelligent processing at a higher degree of automation. Secondly, DL- based descriptive reasoning along a time line can support incremental progressive activity recognition and assistance as an ADL unfolds. The two levels of abstraction in activity modelling, concepts and instances, also allow coarse-grained and fine-grained activity assistance. Thirdly, as the ADL profile of an inhabitant is essentially a set of instances of ADL concepts, it provides an easy and flexible way to capture a user’s activity preferences and styles, thus facilitating personalised ADL assistance. Finally, the unified modelling, representation and reasoning for ADL modelling, recognition and assistance makes it natural and straightforward to support the integration and interoperability between contextual information and ADL recognition. This will support systematic coordinated system development by making use of seamless integration and synergy of a wide range of data and technologies. 

Compared with logic-based approaches, ontology-based approaches have the same mechanisms for activity modelling and recognition. However, ontology-based approaches are supported by a solid technological infrastructure that has been developed in the semantic Web and ontology-based knowledge engineering communities. Technologies, tools and APIs are available to help carry out each task in the ontology-based approach, e.g., ontology editors for context and activity modelling, web ontology languages for activity representation, semantic repository technologies for large-scale semantic data management and various reasoners for activity inference. This gives ontology-based approaches huge advantage in large-scale adoption, application development and system prototyping.

% Finish with advantages and drawbacks of these approaches
\subsection{Summary of knowledge-driven approaches}
The advantages and drawbacks of knowledge-driven approaches are summarised in the following. A summary and comparison with data-driven approaches can be found in Table \ref{tab:soa:comparison}.

\subsubsection*{Advantages}
\begin{itemize}
 \item No ``cold start'' problem: knowledge-driven approaches model activities using generic knowledge rather than data, so activity models are built before deployment and the system does not need any training/learning process before beginning to work.
 \item Interoperability and reusability: specially true for ontology-based approaches, but also for all the other approaches, as activity models are modelled using knowledge engineering approaches and built models are generic and not specific to a concrete user. 
 \item Clear semantics: activity models are semantically clear and can be understood by human beings. This allows interpreting how the system works and developing easier auxiliary systems that work on top of the activity recognition system, such as notification systems, recommender systems, etc. 
\end{itemize}

\subsubsection*{Disadvantages}
\begin{itemize}
 \item Weak in handling uncertainty: inference and reasoning are usually based on certain facts, rather than uncertain sensor information. There are some approaches that work using fuzzy logics and/or probabilistic reasoning, but they are not fully integrated with modelling techniques yet \cite{Helaoui2013}.
 \item Weak in handling temporal information: inference and reasoning mechanisms used for activity recognition do not usually consider temporal aspects of activities. In order to tackle this limitation, there are already some approaches that, for example, integrate ontological and temporal knowledge modelling formalisms for activity modelling \cite{Okeyo2012}. 
 \item Static activity models: knowledge-based activity models are static, since once they are defined, they cannot automatically evolve. This means that if a user changes its way of performing activities, initially defined activity models will still be used for activity recognition.
\end{itemize}


% Insert comparative table here
\newcommand{\specialcell}[2][c]{%
  \begin{tabular}[#1]{@{}c@{}}#2\end{tabular}}

%\begin{table}[htbp]%\tiny
\begin{sidewaystable}[htbp]\scriptsize
    \begin{center}    
        \begin{tabular}{|c|c|c|c|c|c|c|}
            \hline            
            \multicolumn{1}{|c|}{} & \multicolumn{4}{c|}{\textbf{Knowledge-Driven Approaches}} & \multicolumn{2}{c|}{\textbf{Data-Driven Approaches}} \\
            \cline{2-7}
            \multicolumn{1}{|c|}{} & Mining-based & Logic-based & \multicolumn{2}{c|}{Ontology-based} & Generative & Discriminative \\             
            \hline
            \textbf{Model Type}   & \specialcell{HMM, DBN,\\SVM, CRF, NN} & \specialcell{Plans, lattices,\\event trees} & \specialcell{HMM, DBN,\\SVM, CRF, NN} & \specialcell{Sensor and\\Activity ontologies} & \specialcell{Na\"ive Bayes, HMM,\\LDS, DBN} & \specialcell{NN, SVM, CRF,\\Decision tree}  \\
            \hline
	    \specialcell{\textbf{Modelling}\\\textbf{Mechanism}} & \specialcell{Information\\retrieval and\\analysis} & \specialcell{Formal knowledge\\modelling} & \specialcell{(un)supervised\\learning from\\datasets} & \specialcell{Ontological\\engineering} & \specialcell{(un)supervised\\learning from\\datasets} & \specialcell{(un)supervised\\learning from\\datasets}\\
	    \hline
	    \specialcell{\textbf{Activity}\\\textbf{Recognition}\\\textbf{Method}} & \specialcell{Generative or\\discriminative\\methods} & \specialcell{Logical inference\\(deduction, induction)} & \specialcell{Generative or\\discriminative\\methods} & \specialcell{Semantic reasoning\\(subsumption, consistency)} & \specialcell{Probabilistic\\classification} & \specialcell{Similarity or rule\\based reasoning}\\
	    \hline
	    \textbf{Advantage} & \specialcell{No ``cold start''\\problem,\\Using multiple\\data sources} & \specialcell{No ``cold start''\\problem, clear\\semantics on modelling\\\& inference} & \specialcell{Shared terms,\\interoperability\\and reusability} & \specialcell{No ``cold start'' problem,\\multiple models, clear\\semantics on modelling \&\\inference, interoperability\\\& reusability}  & \specialcell{Modelling\\uncertainty,\\temporal\\information} & \specialcell{Modelling\\uncertainty,\\temporal\\information,\\heuristics} \\
	    \hline
	    \textbf{Disadvantage} & \specialcell{The same\\problem as\\DDA} & \specialcell{Weak in handling\\uncertainty and\\scalability}  & \specialcell{The same\\problem as\\DDA} & \specialcell{Weak in handling\\uncertainty and time} & \specialcell{``Cold start''\\problem, lack\\of reusability \&\\scalability} & \specialcell{``Cold start''\\problem, lack\\of reusability \&\\scalability} \\
            \hline
        \end{tabular}
        \caption{The summary and comparison of activity recognition approaches.}
        \label{tab:soa:comparison}
    \end{center}
\end{sidewaystable}