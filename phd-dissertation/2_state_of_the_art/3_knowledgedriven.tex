\section{Knowledge-Driven Approaches}
\label{sec:soa:knowledgedriven}

Knowledge-driven activity modeling is based on real world observations that the list of objects and functionalities to perform an activity are always very similar. For example, to prepare a coffee a liquid container is needed alongside with some coffee and sugar. Even though different people may use different coffee brands, some may add milk and some may prefer brown sugar to white sugar, there are some essential concepts that are always present for every activity. The idea is to be able to use this prior knowledge in order to be able to recognize activities. The implicit relationships between activities, related temporal and spatial context and the entities involved (objects and people) provide a diversity of hint and heuristics for inferring activities.

The first step for knowledge-driven systems is to acquire the needed contextual knowledge. This is usually achieved using standard knowledge engineering approaches. Afterwards, knowledge structures have to be modeled using some of the available ways. For example, schemas, rules or networks. Depending on how the knowledge is captured, different approaches can be distinguished.

Some authors use logic-based approaches for activity recognition. For example, Bouchard et al. \cite{Bouchard2006} map activity recognition to the plan recognition problem in the well studied artificial intelligence field. Another thread of work is to use Event Calculus (EC) techniques, where domains are formalized by fluents, events and predicates. Chen et al. \cite{Chen2008} developed an EC-based framework for behavior reasoning and assistance in a smart home. 

Ontology-based approaches are tightly related to logic-based approaches, but present some interesting differences. Mainly, ontologies allow to adopt a commonly agreed explicit representation of activity definitions which is independent of algorithmic choices, thus facilitating portability, interoperability and reusability. Riboni et al. provide a very good insight about ontology-based activity recognition in \cite{Riboni2011}. They argue that OWL2 is a big step forward from OWL in the sense that it allows reasoning schemes that overcome those offered by OWL. They show an approach to take advantage of OWL2 for activity recognition in another work \cite{Riboni2011a}. 

On the other hand, Chen et al. \cite{Chen2012a} show another approach to use ontologies and semantic reasoning for activity recognition. Their system is able to perform real-time incremental activity recognition, providing coarse-grain (\textit{'MakeDrink'}) and fine-grain (\textit{'MakeCoffee'}) activity recognition. 

Some of the drawbacks of ontology-based approaches are that ontologies are not well suited for temporal and uncertainty modeling. Temporal reasoning is specially important when considering interleaved activities. Okeyo et al. \cite{Okeyo2012} present a way to expand ontology-based systems using semantic rules for temporal reasoning.

In general, knowledge-driven approaches have very interesting features, such as (i) avoiding the cold-start problem by means of modeling activities from knowledge acquired by experts rather than from data, (ii) the generic nature of the approach, since activities are defined by their intrinsic features and not by how certain people perform them and (iii) making systems that are semantically clear and understandable for human beings.

However, there are still some problems to overcome. As mentioned above, even though some logic-based approaches already provide some tools for temporal reasoning, knowledge-driven approaches are in general weak in handling time and uncertainty. Besides, generic activity modeling makes hard to capture peculiarities from different users, which is desirable in order to provide a personalized treatment. 