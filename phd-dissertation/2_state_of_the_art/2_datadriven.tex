\section{Data-Driven Approaches}
\label{sec:soa:datadriven}

The idea behind this approach is to use data mining and machine learning techniques to learn activity models. It is usually presented as a supervised learning approach, where different techniques have been used to learn activities from collected sensor data. Data-driven approaches need big data bases of labeled activities to train different kinds of classifiers. The learning techniques used in the literature are broad, going from simple Naive Bayes classifiers to more complex approaches where several classifiers are combined to recognize activities.

Some representative examples of different supervised learning approaches may begin with the work by Bao et al. \cite{Bao2004}, who use a Naive Bayes classifier for activity recognition based on labeled acceleration data. However, one of the most widely used techniques in the activity recognition community are the Hidden Markov Models (HMM), because activities have a strong relationship with time. Time dependencies and state transitions are well captured by HMMs, hence they tend to perform very well for activity recognition. That was already shown in 1999 by Galata et al. in \cite{Galata1999}, where they show how to use HMMs to learn structured activities. A natural extension of HMMs are the Dynamic Bayesian Networks (DBN). There are many examples of DBN usage for activity recognition: from pedestrian motion monitoring \cite{Brand1997} to office activity monitoring \cite{Oliver2004}. 

HMMs and DBNs have some limitations though. In general, they are not able to capture long-range or transitive dependencies and they need a lot of training examples to recognize accurately some activities. To overcome such problems, there are some other techniques. It is well known that Support Vector Machines (SVM) tend to perform very well with \textit{medium size} data bases. The reason is that SVMs model the boundaries of a class using some special training examples known as "support vectors". So as long as those boundary samples are in the data base, SVMs can model activities well. A recent work by Brdiczka \cite{Brdiczka2009} shows how to use SVMs to recognize human activities in a smart home. 

Another alternative found in the literature is to use Conditional Random Fields (CRF). CRFs are more flexible than HMMs and as many activities may have non-deterministic natures and can be performed in any order, being able to capture those structures is very important. A comparison between HMMs and CRFs for activity recognition can be found in \cite{Vail2007}. Liao et al. present a more realistic application of CRFs to recognize activities based on GPS data in \cite{Liao2007}. 

Data-driven activity recognition has shown to be a good approach. However, it has also some disadvantages. First of all, labeled data bases are needed to train the machine learning techniques used. Obtaining such data bases is very expensive in the context of activity recognition, since complex deployments are usually required. Furthermore, several ethical issues arise because persons are needed to launch experiments in order to collect those data bases. Finally, some experts are required to interpret all the sensor inputs and label them with the correct activity. The cost of running the whole process is high.

Rashidi and Cook show how to overcome the problem of manually labeling activity data bases in \cite{Rashidi2011}. They use a non-labeled data base, where they extract activity clusters using non-supervised learning techniques. Those clusters are used to train a boosted HMM, which is shown to be able to recognize several activities. 

However, the work by Rashidi and Cook is not able to overcome some other traditional problems of data-driven approaches. Mainly, data-driven approaches suffer the cold-start problem, i.e. they need to collect a lot of data before they can start working. Besides, data-driven approaches have many difficulties when generalizing what they have learned from previous users, since every user has his/her own ways of performing activities. Those problems suggest the need of a different approach. That approach is the knowledge-driven activity recognition approach.