\section{Summary and Conclusions}
\label{sec:soa:summary}

%Put a table like the one we can find in Chen's survey paper, showing the strengths and weaknesses of each approach. Highlight which are the topics that are not covered in the literature and are addressed in this dissertation.

A deep review of human activity recognition has been provided in this chapter. In terms of activity monitoring technologies, the work presented in this dissertation can be classified into the dense sensing-based category. In principle, there are no limitations to extend the methods and techniques exposed in this dissertation to other activity monitoring approaches such as wearable sensor- or vision-based categories. But dense sensing paradigm has been chosen because it is the best approach for Intelligent Environments - since there are no privacy issues with users and specific objects can be monitored - and the usage of simple sensors makes sensor processing steps simpler. As research on sensor processing is beyond the scope of this work, dense sensing-based activity monitoring provides a perfect scenario. Notice that the selection of the dense sensing-based activity monitoring scenario is more an implementation decision rather than a theoretical limitation.

In terms of activity modelling, the work presented in this dissertation clearly falls into the hybrid approach. It combines knowledge- and data-driven techniques in order to provide dynamic activity models that combine generic and personalised models. As such, it tries to solve the problem of the hybrid activity modelling approach presented by Chen et al. \cite{Chen2014}, i.e. learning new actions for any user. Solving other problems of knowledge-driven approaches, such as sensor uncertainty or temporal information handling, is out of scope of this dissertation. The concrete position of this dissertation in the two axis taxonomy adopted for activity recognition can be seen in Table \ref{tab:soa:classification}.

\begin{table}[htbp]\small
    \begin{center}    
        \begin{tabular}{|c|c|c|c|c|}
            %\hline
            \cline{3-5}
            \multicolumn{2}{c}{} & \multicolumn{3}{|c|}{\textbf{Activity Monitoring Approach}} \\
            \cline{3-5}
            \multicolumn{2}{c}{} & \multicolumn{1}{|c|}{Vision-based} & \multicolumn{2}{|c|}{Sensor-based} \\
            \cline{4-5}
            \multicolumn{2}{c}{} & \multicolumn{1}{|c|}{} & Wearable-based & Dense sensing-based \\
            \hline
            \multirow{3}{*}{\specialcell{\textbf{Activity}\\\textbf{Modelling}\\\textbf{Approach}}} & Data-Driven & & & \\
            \cline{2-5}
             & Knowledge-Driven & & &\\
            \cline{2-5}
             & Hybrid & & & \textbf{X}\\
            \hline
        \end{tabular}
        \caption{The classification of this dissertation in terms of activity modelling and activity monitoring approaches marked with an \textbf{X}.}
        \label{tab:soa:classification}
    \end{center}
\end{table}
        

Finally, this chapter has been mainly focused on single user - single activity scenarios, although some examples of single user - concurrent scenarios were also described. This is because the human activity recognition research community has mainly focused its efforts in the single user - single activity scenario, despite recently, single user - concurrent activities scenario has gained more popularity. It is worth to highlight that the work presented in this dissertation has been designed to tackle the single user - single activity scenario. 

