\section{Summary and Conclusions}
\label{sec:soa:summary}

%Put a table like the one we can find in Chen's survey paper, showing the strengths and weaknesses of each approach. Highlight which are the topics that are not covered in the literature and are addressed in this dissertation.

A deep review of human activity recognition has been provided in this chapter. In terms of activity monitoring technologies, the work presented in this dissertation can be classified into the dense sensing based category. In principle, there are not limitations to extend the methods and techniques exposed in this dissertation to other activity monitoring approaches such as wearable sensor based or vision-based categories. But dense sensing paradigm has been chosen because it is the best approach for Intelligent Environments and the usage of simple sensors makes sensor processing steps simpler. As research on sensor processing is out of the scope of this work, dense sensing based activity monitoring provides a perfect scenario. The selection of the dense sensing based activity monitoring scenario is more an implementation decision rather than a theoretical limitation.

In terms of activity modelling, the work presented in this dissertation clearly falls into the hybrid approach. It combines knowledge- and data-driven techniques in order to provide dynamic activity models that combine generic and personalised models. As such, it tries to solve the problem of the hybrid activity modelling approach presented by Chen et al. \cite{Chen2014}, i.e. learning new actions for any user. However, it is worth highlighting that solving the other problems of knowledge-driven approaches, i.e. sensor uncertainty and temporal information handling is out of scope of this dissertation.

Finally, this chapter has been mainly focused on single user - single activity scenarios, although some examples of single user - concurrent scenarios were also described. This is because human activity recognition research community has focused its efforts in the single user - single activity scenario. Recently, single user - concurrent activities scenario is gaining more popularity. Nevertheless, this dissertation tackles the single user - single activity scenario. 