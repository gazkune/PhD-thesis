
% this file is called up by thesis.tex
% content in this file will be fed into the main document

%: ----------------------- introduction file header -----------------------
\begin{savequote}[50mm]
Write here a quote
\qauthor{Quote author}
\end{savequote}


\chapter{The Approach to Learn Specialised and Complete Activity Models}
\label{cha:archi}

% the code below specifies where the figures are stored
\ifpdf
    \graphicspath{{3_approach_to_learning_eam/figures/PDF/}{3_approach_to_learning_eam/figures/PNG/}{3_approach_to_learning_eam/figures/}}
\else
    \graphicspath{{3_approach_to_learning_eam/figures/EPS/}{3_approach_to_learning_eam/figures/}}
\fi

\letra{T}{his} chapter describes the general architecture of the solution designed to learn activity models from user's behavioural data and previous knowledge. Provide here also the basis of ontology-based activity modelling to provide a solid base for our approach. Show the intuition that led us to the final solution (the plot with actions in three axis: time, location and type). Provide the design architecture, implementation details and finally the definitions and constraints related to the solution.

Here, show how the chapter is structured in sections:

\begin{itemize}
 \item Ontology-Based Activity Modelling: describe how ontology-based activity modelling works, based on Chen's papers. Section \ref{sec:approach:ontology}.
 \item Definitions and Constraints: provide accurate definitions of all relevant concepts in the dissertation. Establish clearly the constraints of the work (single-user single-activity scenario, dense sensing activity monitoring...). Section \ref{sec:approach:def}.
 \item Solution Design: describe the high-level solution, module by module; provide implementation details of every module and reference where the module is described in this dissertation; describe the inputs of the system (context knowledge, sensor activation dataset) and the outputs (partially annotated dataset, fully annotated dataset, learnt action patterns...). Section \ref{sec:approach:solution}.
\end{itemize}

