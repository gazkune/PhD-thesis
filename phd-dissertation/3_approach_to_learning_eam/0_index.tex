
% this file is called up by thesis.tex
% content in this file will be fed into the main document

%: ----------------------- introduction file header -----------------------
\begin{savequote}[50mm]
Simplicity is the shortest path to a solution.
\qauthor{Ward Cunningham}
\end{savequote}


\chapter{The Approach to Learn Specialised and Complete Activity Models}
\label{cha:archi}

% the code below specifies where the figures are stored
\ifpdf
    \graphicspath{{3_approach_to_learning_eam/figures/PDF/}{3_approach_to_learning_eam/figures/PNG/}{3_approach_to_learning_eam/figures/}}
\else
    \graphicspath{{3_approach_to_learning_eam/figures/EPS/}{3_approach_to_learning_eam/figures/}}
\fi

\letra{T}{his} chapter describes the theoretical foundations of the dissertation and describes the high-level solution designed to learn activity models from user's behavioural data and previous knowledge. The contributions of this dissertation are based on the ontology-based activity modelling approach, where an important limitation has been found: providing complete and generic activity models is generally impossible. To solve this problem, a learning solution is suggested, whose underpinning modules and their relationships are described in detail. The integration of the developments presented in Chapters \ref{cha:clustering} and \ref{cha:learner} is structured around the designed solution in this chapter.

%Provide here also the basis of ontology-based activity modelling to provide a solid base for our approach. Show the intuition that led us to the final solution (the plot with actions in three axis: time, location and type). Provide the design architecture, implementation details and finally the definitions and constraints related to the solution.

The chapter is divided in four sections: Section \ref{sec:approach:ontology} describes the ontology-based activity modelling approach on which the contributions of this dissertation are built. Section \ref{sec:approach:def} provides formal definitions of the concepts and terms that will be used during the following chapters and shows the constraints of the approach to learn activity models. Section \ref{sec:approach:solution} discusses the adopted solution design, describing the defined architecture, constituent modules and their relationships. And finally Section \ref{sec:approach:sum} concludes the chapter with a summary and extracted conclusions.