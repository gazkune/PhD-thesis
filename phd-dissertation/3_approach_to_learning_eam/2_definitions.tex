\section{Definitions and Constraints}
\label{sec:approach:def}

For the rest of the dissertation the following terms and concepts will be used according to the definitions provided in this section:

\begin{defn}[Sensor activation (SA)]
\label{def-sa}
 A sensor activation occurs when a sensor changes its state from the no-interaction state to interaction state. Reverse transitions, i.e. de-activations of sensors, are not considered. For example, when a user takes a glass, the activation is tagged as \textit{glassSens}. A sensor activation is then composed by a time-stamp and a sensor identifier:
 
 \begin{equation}
  SA = \{ time-stamp, sensorID \}
 \end{equation}

\end{defn}

\begin{defn}[Sensor activation dataset]
\label{def-sa-dataset}
 A time-ordered sequence of sensor activations.
\end{defn}

\begin{defn}[Actions]
\label{def-action}
 Actions are the primitives of activities and are directly linked to sensor activations. For example, the activation of sensors \textit{cupSens} and \textit{glassSens} are linked to the action \textit{hasContainer}. A sensor activation can only be mapped to one single action, but an action can be generated by different sensor activations. Actions do not have any duration and are hence described by a time-stamp.
\end{defn}

\begin{defn}[Type]
\label{def-type}
 When referring to activities and objects, type captures a purpose based classification. In this dissertation, considered types are \textit{Cooking}, \textit{Hygiene}, \textit{Entertainment} and \textit{Housework}. An activity can only have one type, while objects can have more than one. For example, an object like a tap can be used for cooking, housework or hygiene. But a MakeCoffee activity can only be a cooking activity. On the other hand, when referring to sensors, type classifies sensors in terms of their technological base. In this dissertation, modelled sensor types are \textit{contact}, \textit{electric}, \textit{pressure} and \textit{tilt} sensors. 
\end{defn}

\begin{defn}[Location]
\label{def-location}
 Location refers to semantic places of the monitored environment, where sensor activations occur and hence, activities are performed. For example, in a house, possible locations might be \textit{bedroom}, \textit{bathroom} or \textit{kitchen}. 
\end{defn}

\begin{defn}[Initial Activity Model (IAM)]
\label{def-iam}
Activity models are sequences of actions defined by a domain expert. Initial activity models refer to the minimum number of necessary actions to perform an activity. The objective of such models is to represent incomplete but generic activity models applicable to any user. Initial activity models also have an estimation of the maximum duration of the activity.
\begin{equation}
  IAM(Activity_n) = \{action_a, action_b, \ldots , max\_duration\}
 \end{equation} 
\end{defn}

\begin{defn}[Extended Activity Model (EAM)]
\label{def-eam}
 A complete and specialised version of an IAM. By \textit{complete} we mean that an activity model contains all the actions performed by a user for the corresponding activity. By \textit{specialised} we mean there are two or more different complete action sequences for the corresponding activity, i.e. specialised sub-classes of that activity exist. The EAM for an activity is represented as a list of action sequences with their occurrence frequency:
\begin{equation}
 \begin{split}
 EAM(Activity_n) = \{as_1, as_2, \ldots , as_n\} \ \ \ where \\
 as_i = \{frequency, action_a, action_b, \ldots, action_m\}
 \end{split}
 \label{eq-eam}
\end{equation}
\end{defn}

The learning approach for activity models presented in this dissertation has some constraints:

\begin{cons}[Dense sensing-based activity monitoring]
\label{cons-dense}
 Even though there is no theoretical constraint that prevents the approach to be implemented for other activity monitoring approaches, the implementation presented in this dissertation is designed for the dense sensing-based activity monitoring approach.
\end{cons}

\begin{cons}[Single user - single activity scenario]
\label{cons-single}
 This is a hard constraint, in the sense that the theoretical foundations of the learning approach assume explicitly that only one user is being monitored and this user is not allowed to perform concurrent or interleaving activities.
\end{cons}

\begin{cons}[Uniquely located activities]
\label{cons-location}
 It is assumed that an activity can only be performed in a single location, during a concrete execution, i.e. although ReadBook can be performed in the bedroom and in the lounge, a concrete execution will take place in the bedroom or in the lounge (exclusive or).
\end{cons}

\begin{cons}[``Static'' objects]
\label{cons-static-obj}
 Objects used by a user are assumed to be ``static'' respect to the location, i.e. if the location of a brusher is the bathroom, the object will always stay in the bathroom.
\end{cons}