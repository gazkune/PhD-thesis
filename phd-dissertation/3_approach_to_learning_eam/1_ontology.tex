\section{Ontology-Based Activity Modelling}
\label{sec:approach:ontology}

The approach presented in this paper is based on the dense sensing activity monitoring paradigm, which establishes the idea of inferring activities by monitoring human-object interactions through the usage of multiple multi-modal miniaturised sensors. Single-user single-activity scenarios are considered, where only one user is monitored and concurrent activities cannot be performed.

In this scenario, ontology-based activity recognition systems have shown to perform robustly \cite{Chen2012a}. Central to those approaches is the ontology-based activity modelling. Activities are defined as ontological concepts and all actions that are required to perform the activity as the properties of the concept. For example, making tea involves taking a cup from the cupboard, putting a teabag into the cup, adding hot water to the cup, then milk and/or sugar. The ontological model of making tea, i.e. \textit{MakeTea} concept, can be defined by action properties \textit{hasContainer}, \textit{hasTeabag}, \textit{hasHotwater}, \textit{hasMilk} and \textit{hasFlavour} in conjunction with descriptive properties such as activity start time \textit{actStartTime} and duration \textit{actDuration}.

Activities can be modelled at different levels of abstraction. As such, ontological activity concepts are usually organised in a hierarchical structure to form super-class and sub-class relationships. For example, \textit{MakeTea}, \textit{MakeCoffee} and \textit{MakeHotChocolate} activities can be modelled as the subclasses of \textit{MakeHotDrink} activity, which is in turn the subclass of \textit{MakeDrink}. 

Ontology-based activity modelling provides semantically clear, structured and reusable models. Furthermore, it offers a unified framework to combine generic models that can be applied to any user with personal models. However, the main problem of this modelling approach is that obtaining complete models for every person is not generally possible, because even though there are certain actions that every user performs for a given activity, there might be some other actions that cannot be known in the modelling step.