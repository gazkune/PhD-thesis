\section{Summary and Conclusions}
\label{sec:approach:sum}

A global solution has been designed to learn extended activity models from initial generic but incomplete activity models. The base of the solution is the ontology-based activity modelling, which provides a unified and reusable framework to model ADLs. The presented solution for learning aims at completing and specialising initial activity models provided by a domain expert. 

For that purpose, this chapter introduces some important definitions and constraints, setting the scenario in which the learning system works. Afterwards, the rationale behind the solution design has been described. It has been shown that when actions pertaining to activities are plotted in the so called activity space - spanned by time, type and location - activities are arranged in clusters. The idea is to be able to identify those clusters and aggregate surrounding actions defining appropriate metrics and heuristics. Once those clusters have been identified, a learning algorithm will be deployed to learn extended activity models from identified clusters.

In conclusion, this chapter has established the ground in which the contributions of the dissertation stand. The addressed problem has been described and a solution for that problem has been proposed. The following chapters will show how this solution is deployed.