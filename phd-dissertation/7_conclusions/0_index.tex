
% this file is called up by thesis.tex
% content in this file will be fed into the main document

%: ----------------------- introduction file header -----------------------
\begin{savequote}[50mm]
Alea jacta est.
\qauthor{Gaius Julius Caesar}
\end{savequote}


\chapter{Conclusions and Future Work}
\label{cha:conclusions}

% the code below specifies where the figures are stored
\ifpdf
    \graphicspath{{7_conclusions/figures/PDF/}{7_conclusions/figures/PNG/}{7_conclusions/figures/}}
\else
    \graphicspath{{7_conclusions/figures/EPS/}{7_conclusions/figures/}}
\fi

\letra{A}{} general description of the main results and contributions presented in this dissertation is given in this chapter. To finalise the dissertation, the objectives posed in Chapter \ref{cha:introduction} are reviewed to see at what extent they have been achieved. Furthermore, a summary of all contributions done during the research work is provided, alongside a list of publications which prove how this work has been validated with the research community. The chapter ends with some ideas for future research in the area of activity modelling and some final remarks.

The rest of this chapter is divided in the following sections: Section \ref{sec:conclusions:conclusions} presents a summary of work and the conclusions obtained from the research work and results. Section \ref{sec:conclusions:contrib} lists all the contributions done through the dissertation. Section \ref{sec:conclusions:hypothesis} shows how the objectives and goal of the dissertation are achieved, thus validating the hypothesis with the obtained results. Section \ref{sec:conclusions:pub} reviews all relevant scientific publications related to the development of the dissertation. Section \ref{sec:conclusions:future} provides some ideas for future research, extracted from the limitations and weaknesses of the EAM learning system. Section \ref{sec:conclusions:final} makes some final remarks.