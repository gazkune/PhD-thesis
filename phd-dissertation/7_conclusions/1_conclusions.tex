%\section{Conclusions}
\section{Summary of Work and Conclusions}
\label{sec:conclusions:conclusions}

Activity modelling is a very important step for activity recognition. It provides the models which will be used in the recognition step and thus encode the exact way an activity is performed. There are several desirable features for activity modelling approaches, including:

\begin{enumerate}
 \item Provide generic activity models which can be applied to any user.
 \item Provide personalised activity models which capture the special way an activity is performed by a concrete user.
 \item Provide tools to make activity models evolve in time as users vary their behaviour.
 \item Provide human-understandable models, to make the usage of such models easy for other applications for intelligent systems.
\end{enumerate}

With the objective of achieving an activity modelling process which fulfils all the listed requirements, a general framework was presented in Chapter \ref{cha:introduction}, represented in Figure \ref{fig-activity-modelling}. As a summary, the process starts with an expert who provides generic activity models. Those models are used in the learning system to identify activities in sensor datasets and learn personalised models. The expert reviews the obtained personalised models and includes them in the knowledge base. Repeating those steps continuously in time, dynamic activity models are obtained.

One of the main gaps to achieve the proposed activity modelling process was to learn new actions for already known activities. So the objective of the dissertation was to fill in that gap, combining knowledge-driven activity models with data-driven learning techniques. The models learned would be personal activity models with specific action sequences per user obtained from the sensor datasets generated while performing activities.

This dissertation has introduced a novel approach to acquire complete and specialised knowledge-driven activity models through data-driven learning techniques. The approach makes possible using generic but incomplete knowledge-driven activity models to learn specialised and complete models, which represent the personal way an activity is performed by a concrete user. Central to the approach is the two-step clustering process (Chapter \ref{cha:clustering}), which has the singularity of using previous knowledge in order to find activity clusters and label them with the appropriate activity class. Those clusters are then treated by the learning algorithm in order to acquire extended activity models for different users (Chapter \ref{cha:learner}).

The results as shown in Chapter \ref{cha:evaluation} indicate that the approach works well in realistic experimental set-ups, with real users' inputs, realistic time intervals for activity sequences and sensor noise. It has to be stressed that all the varying ways of performing an activity by a user are correctly captured by the approach with a success rate of 100\% for all users and scenarios. Specialised and complete models for initial generic incomplete activity models can be properly learned automatically with minimum previous context knowledge. In consequence, the presented approach can be used to make knowledge-driven activity models evolve with user behavioural data, offering the tools to solve two of the problems of knowledge-driven approaches: the generic and static nature of activity models.

So it can be concluded that the EAM learning system developed in this dissertation has served to validate the hypothesis posed in Section \ref{sec:intro:hypothesis}. It has been shown that personalised activity models can be learned accurately. Accuracy, in this case, refers to the ability to learn all the actions executed by a user to perform a concrete activity. The 100\% rate of true positives achieved in the experiments backs the statement.

The evaluation methodology used relies on simulation tools and user surveys (Chapter \ref{cha:evaluation}). There are some limitations, since users might omit some details in their answers and the synthetic dataset generator cannot accurately simulate all the possible situations. For example, simulating user erratic behaviour is a challenge. That is why the level of noise introduced in the experiments is so high (around 38\% in average). The idea is to introduce high levels of positive sensor noise, much higher than those seen in real pervasive environments (see \cite{Chen2012}). Combining real users' inputs regarding activities, objects, time lapses and locations, realistic sensor error models obtained from real environments and high levels of positive sensor noise to properly cover the effects of user erratic behaviour, the results obtained using this evaluation methodology can be deemed as relevant. The most important thing is to be able to capture what users describe in their surveys, showing that new actions can be learned and different ways of performing the same activity can be identified and properly modelled. The evaluation methodology designed and described in Section \ref{subsec:evaluation:hybrid} guarantees this. Nevertheless, a final validation on real data would be desirable. Despite the technical difficulties of running such experiments, we are currently working on it.

In conclusion, the EAM learning system designed and developed through this dissertation, is composed by the activity clustering process (Chapter \ref{cha:clustering}) and the Activity Model Learner ($AML$, Chapter \ref{cha:learner}). The activity clustering process is divided into two steps: the initialisation of the clustering, provided by the Semantic Activity Annotation Algorithm ($SA³$, Section \ref{sec:clustering:sa3}) and the Action Aggregator algorithm ($AA$, Section \ref{sec:clustering:ac}). This learning system has been designed to cope with the problem of learning personalised activity models in terms of actions, using generic but incomplete activity models provided by a domain expert. Such a problem has been identified to be one of the gaps to achieve an activity modelling process for dynamic and personalised models. Once results are carefully analysed, it can be claimed that the EAM learning system accomplishes its objectives, being able to learn extended activity models for several users. So the EAM learning system is an important step towards a dynamic and personalised knowledge-driven activity modelling process. 
