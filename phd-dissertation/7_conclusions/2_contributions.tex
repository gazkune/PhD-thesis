\section{Contributions}
\label{sec:conclusions:contrib}

A summary of the contributions explained in this dissertation is presented in this section:

\begin{itemize}
 \item An in-depth state of the art for activity monitoring, modelling and recognition was presented in Chapter \ref{cha:soa}.
 \begin{itemize}
  \item Providing a detailed taxonomy of activity recognition approaches.
  \item Analysing the advantages and disadvantages of activity modelling approaches.
 \end{itemize}

 \item A knowledge representation formalism for activities and intelligent environments was presented in Chapter \ref{cha:archi}.
 \begin{itemize}
  \item It offers a light-weight knowledge representation and processing based on JavaScript Object Notation (JSON) technology.
  \item It uses compact and easy-to-extend structures for activities, objects, locations and sensors, to achieve environment independence.
 \end{itemize}

 \item The design architecture for extended activity models was described in Chapter \ref{cha:archi}.
 \begin{itemize}
  \item It offers a convenient modular architecture which separates key concepts and allows the implementation of different approaches.
 \end{itemize}

 \item A novel activity clustering process which uses previous knowledge in unlabelled sensor datasets is explained in Chapter \ref{cha:clustering}.
 \begin{itemize}
  \item The initialisation step is provided by $SA³$, which uses initial activity models to run a specific pattern recognition algorithm and find activity clusters.
  \item New actions are aggregated to initial clusters using the $AA$ algorithm, based on location, type and time analysis of actions and activities.
  \item The activity clustering algorithm is a good example of how domain knowledge can be used to form semantically meaningful clusters in unlabelled datasets.
 \end{itemize}

 \item An activity model learning algorithm was presented in Chapter \ref{cha:learner}, which uses activity clusters to extract activity models.
 \begin{itemize}
  \item It has a flexible filtering step to pre-process the activity clusters in form of action sequences found by the activity clustering algorithm.
  \item It implements a model similarity-based outlier detection statistical algorithm to detect spurious activity models.
  \item It outputs complete and specialised activity models for a concrete user alongside with their occurrence frequencies.
 \end{itemize}

 \item A hybrid evaluation methodology was designed and implemented to properly evaluate the learning system in Chapter \ref{cha:evaluation}. The methodology can also be used for further activity modelling and recognition approaches.
 \begin{itemize}
  \item It is based on surveys to users in order to capture how activities are performed by them.
  \item Using surveys, it allows obtaining the information of any number of users minimising time-cost.
  \item A special simulator is implemented: the synthetic dataset generator. Its usage allows generating on demand any kind of dataset.
  \item It allows modelling sensor errors.
  \item Combining surveys and simulation tools, any kind of experiment can be prepared generating as much data as needed.
 \end{itemize}

\end{itemize}
