\section{Objectives}
\label{sec:learner:objectives}

%Explain the objective of learning extended activity models: complete and specialised models.
The main objective of the Activity Model Learner is to learn extended activity models (EAM), which have been defined in definition \ref{def-eam}. An EAM of an activity represents complete and specialised models of the activity for a concrete user. Assuming that different users execute different actions to perform the same activity and that a concrete activity may be performed in different ways by the same user, a user adaptable activity modelling approach has to be able to capture activity models that fulfil these requirements. For example, making a coffee is a common activity for many users. However, a concrete user may sometimes make a coffee with milk and other times black coffee. Both, coffee with milk and black coffee, are specialised sub-activities of making coffee, which will be characterised by different action sequences. So the objective of $AML$ is to learn those different action sequences for a concrete user based on the output given by the clustering process described in Chapter \ref{cha:clustering}.

\begin{problem}[$AML$]
\label{pro-aml}
 Given the activity clusters and fully annotated dataset files coming from the activity clustering process, learn extended activity models, i.e. complete and specialised activity models for a concrete user.
\end{problem}

Notice that in some cases, a user may only perform an activity in a single way. In that case $AML$ would learn only a complete activity model, since for a model to be considered a specialised model, at least two complete models have to be learned. 

%Explain the role of the expert: learnt models are presented to an expert.
When extended activity models are learned, they are presented to the domain expert, mainly due to two reasons: (i) the specialised models do not have a semantic tag as sub-activities of the detected activity and (ii) to let the expert analyse and add the learned models to the main activity ontology if convenient. The first reason refers to the fact that $AML$ cannot know whether an specialised model refers to making a black coffee or making coffee with milk. Nevertheless, having the specialised activity model, it is usually easy for an expert to decide the semantic label of the specialised activity model. The second reason is derived from the fact that $AML$ will not be able to deliver without any error, specially considering the limited previous knowledge provided and the noisy sensor scenario in which it has to work. An expert will analyse the EAMs learned by $AML$ to filter them if necessary and add them to the activity ontology with appropriate specialised activity labels.

%Explain the objective to learn all the variations actually performed by a user: conservative approach.
Based on the role of the domain expert and assuming that the activity clusters obtained in the clustering process do contain all the activity variations for a concrete user, a conservative approach for activity learning is the best option. In this context, conservative means that it is better to learn false activity models as far as all real activity models are learned, i.e. false positives can be assumed if true positives are 100\%. If an activity variation detected by the clustering process is removed in the learning stage, the expert will no be able to recover that variation from the information provided by $AML$. However, false activity models can be purged by the expert, so the objective of the $AML$ is to keep true positive rates of 100\% minimising the false positive rates. 

