\section{The Learning Algorithm}
\label{sec:learner:algorithm}

Activity clusters extracted by the clustering process contain all the action sequences performed by a user for each activity. But some of those clusters are spurious, due to sensor noise, user erratic behavior and clustering errors. The objective of $AML$ is to remove spurious action sequences. For that purpose, a three-step algorithm has been designed and implemented. For all the clusters extracted for an activity, the following steps are performed:

\begin{enumerate}
 \item Remove repeated actions into a sequence: some action sequences contain repeated actions. For instance, consider the sequence $S=\{a, b, a, c\}$. As repeated actions do not add any new information for activity modeling, they are removed. $S$ becomes $S' = \{a, b, c\}$. Notice that this step does not remove any action sequence of an activity.
 \item Fuse equal action sequences: some action sequences contain the same actions, but in different orders. For example, $S_1 = \{a, b, c \}$ and $S_2 = \{b, c, a\}$. The order of actions is not important for activity models, so both sequences are fused. To detect equal sequences, the Jaccard coefficient is used \cite{A.K.Jain1988}. The Jaccard coefficient between two sequences $A$ and $B$ is defined as:
 \begin{equation}
  Jaccard(A, B) = \frac{A \cap B}{A \cup B}
 \end{equation}
Any two sequences whose Jaccard coefficient is 1 are fused. Fusing means removing one of the sequences and adding frequencies to store in the remaining sequence.
 \item Run Jaccard based outlier detection algorithm, which has been specially designed to learn proper action sequences.
\end{enumerate}

The Jaccard based outlier detection is an iterative algorithm. It calculates the so called \textit{Jaccard Matrix} ($JM$), which is a square matrix of remaining action sequences. $JM_{i, j}$ stores the Jaccard coefficient for action sequences $i$ and $j$. The diagonal of $JM$ is 1, since two equal action sequences' Jaccard coefficient is 1. Removing diagonal values $\hat{JM}$ is obtained, which is used to calculate the median and the standard deviation to the median. The median is used rather than the mean value, since the median is robust to outliers. Using those statistics a threshold $\theta$ is calculated, such that:

\begin{equation}
 \theta = max \{ median(\hat{JM}) + std(\hat{JM}), \lambda \}
\end{equation}

The first part of the calculation of $\theta$ captures the relative similarity among all action sequences and establishes an adequate threshold to identify outliers. However, as the Jaccard coefficient is defined in an absolute scale, the second part ($\lambda \in [0, 1]$) has to be added. For relatively short action sequences used in the experiments (the longest ones are around 9 actions), 0.75 has shown to be a good balanced value. $\lambda$ prevents fusing sequences that even being more similar than most of the others, their similarity is not higher than it. Sequences below $\lambda$ are considered too different to be fused. 

The algorithm fuses sequences whose Jaccard coefficient is higher than $\theta$, until no sequences can be fused. To fuse, the sequence with lowest frequency is removed and its frequency value is added to the frequency of the other sequence. This fusing heuristic states that the lower frequency sequence is a spurious variation of the higher frequency sequence. 